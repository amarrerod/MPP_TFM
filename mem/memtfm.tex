\documentclass[english,a4paper,14pt,oneside]{extreport}

%%%%%%%%%%%%%%%%%%%%%%%%%%%%%%%%%%%%%%%%%%%%%%%%%%%%%%%%%%%%%%%%%%%%%%%%%%%%%%%
\usepackage[official]{eurosym}
\usepackage[dvips]{graphicx}
\usepackage[dvips]{epsfig}
\usepackage[utf8]{inputenc}
\usepackage{cite}
\usepackage{alltt}
\usepackage{multirow}
\usepackage{caption}
\usepackage{amsmath}
\usepackage{booktabs}
\usepackage[table,xcdraw]{xcolor}
\usepackage{rotating,tabularx}
\usepackage{algorithmic}
\usepackage[]{algorithm2e}
\usepackage{afterpage}
\usepackage{amsfonts}
\usepackage{xcolor}
\usepackage{capt-of}% or use the larger `caption` package
\usepackage{amstext} % for \text macro
\usepackage{array}   % for \newcolumntype macro
\newcolumntype{L}{>{$}l<{$}} % math-mode version of "l" column type
\usepackage[top=2cm, bottom=2cm, left=2cm, right=2cm]{geometry}
%%%%%%%%%%%%%%%%%%%%%%%%%%%%%%%%%%%%%%%%%%%%%%%%%%%%%%%%%%%%%%%%%%%%%%%%%%%%%%%

\newcommand{\SONY}{{\sc Sony}}
\newcommand{\MICROSOFT}{{\sc Microsoft}}
\newcommand{\GCC}{\textsf{\textsc{G}CC}}
\newcommand{\INTEL}{\textsf{\textsc{I}ntel}}



%%%%%%%%%%%%%%%%% Creamos un entorno para listar código fuente %%%%%%%%%%%%%%%
\newenvironment{sourcecode}
{\begin{list}{}{\setlength{\leftmargin}{1em}}\item\scriptsize\bfseries}
{\end{list}}

\newenvironment{littlesourcecode}
{\begin{list}{}{\setlength{\leftmargin}{1em}}\item\tiny\bfseries}
{\end{list}}

\newenvironment{summary}
{\par\noindent\begin{center}\textbf{Abstract}\end{center}\begin{itshape}\par\noindent}
{\end{itshape}}

\newenvironment{keywords}
{\begin{list}{}{\setlength{\leftmargin}{1em}}\item[\hskip\labelsep \bfseries Keywords:]}
{\end{list}}

\newenvironment{palabrasClave}
{\begin{list}{}{\setlength{\leftmargin}{1em}}\item[\hskip\labelsep \bfseries Palabras clave:]}
{\end{list}}


%%%%%%%%%%%%%%%%%%%%%%%%%%%%%%%%%%%%%%%%%%%%%%%%%%%%%%%%%%%%%%%%%%%%%%%%%%%%%%%
% Format
%%%%%%%%%%%%%%%%%%%%%%%%%%%%%%%%%%%%%%%%%%%%%%%%%%%%%%%%%%%%%%%%%%%%%%%%%%%%%%%
%\usepackage{showframe}
%\marginparwidth 0mm
%%\topmargin -4 mm
%\topmargin -21 mm
%\headheight 10 mm
%\headsep 10 mm

%\textheight 229 mm
%\textheight 246 mm

%\oddsidemargin -5.4 mm
%\evensidemargin -5.4 mm
%\oddsidemargin 5 mm
%\evensidemargin 5 mm

%\oddsidemargin -3 mm
%\evensidemargin -3 mm

%\textwidth 17 cm
%\textwidth 15 cm
%\columnsep 10 mm

\input{amssym.def}

%%%%%%%%%%%%%%%%%%%%%%%%%%%%%%%%%%%%%%%%%%%%%%%%%%%%%%%%%%%%%%%%%%%%%%%%%%%%%%%

\begin{document}

%%%%%%%%%%%%%%%%%%%%%%%%%%%%%%%%%%%%%%%%%%%%%%%%%%%%%%%%%%%%%%%%%%%%%%%%%%%%%%%
% First Page
%%%%%%%%%%%%%%%%%%%%%%%%%%%%%%%%%%%%%%%%%%%%%%%%%%%%%%%%%%%%%%%%%%%%%%%%%%%%%%%

\pagestyle{empty}
\thispagestyle{empty}


\newcommand{\HRule}{\rule{\linewidth}{1mm}}
\setlength{\parindent}{0mm}
\setlength{\parskip}{0mm}

\vspace*{\stretch{0.5}}

\begin{center}
\includegraphics[scale=1.5]{images/marca.png}\\[10mm]
{\Huge Masters Degree Thesis}
\end{center}

\HRule
\begin{flushright}
        {\Huge On the healthy and balanced menu planning automatisation} \\[2.5mm]
        {\Large Planificación automática de menús saludables y equilibrados}\\[2.5mm]
        {\Large Alejandro Marrero Díaz} \\[5mm]


\end{flushright}
\HRule
\vspace*{\stretch{2}}
\begin{center}
  \Large La Laguna, XX March 2019
\end{center}

\setlength{\parindent}{5mm}

%%%%%%%%%%%%%%%%%%%%%%%%%%%%%%%%%%%%%%%%%%%%%%%%%%%%%%%%%%%%%%%%%%%%%%%%%%%%%%%
% Signature page (add the official stamp)
%%%%%%%%%%%%%%%%%%%%%%%%%%%%%%%%%%%%%%%%%%%%%%%%%%%%%%%%%%%%%%%%%%%%%%%%%%%%%%%
\newpage
%\cleardoublepage
\thispagestyle{empty}

Mr. {\bf Eduardo Manuel Segredo González}, con N.I.F. 78-564-242-Z
Part-time lecturer at Departamento 
de Ingeniería Informática y de Sistemas
de la Universidad de La Laguna, as mentor

\bigskip
Mrs. {\bf Coromoto Antonia León Hernández}, with N.I.F. 78-605-216-W
Full-time lecturer at Departamento 
de Ingeniería Informática y de Sistemas
de la Universidad de La Laguna, as mentor

\bigskip
\bigskip
{\bf C E R T I F Y}

\bigskip
\bigskip
\bigskip
That this report named:

\bigskip
``{\it On the healthy and balance menu planning automatisation.}''

\bigskip
\bigskip
\bigskip

\noindent has been done under they supervision by Mr. {\bf Alejandro Marrero Díaz},
with N.I.F. 78-649-404-F.

\bigskip
\bigskip

And to be recorded, in compliance of the current legislation and timely effects XX March 2019.

%\cleardoublepage
\newpage
%%%%%%%%%%%%%%%%%%%%%%%%%%%%%%%%%%%%%%%%%%%%%%%%%%%%%%%%%%%%%%%%%%%%%%%%%%%%%%%
\thispagestyle{empty}

{ \flushright

\begin{LARGE}
Acknowledgements 
\end{LARGE}

\hspace{3mm}

\begin{large}


\hspace{3mm}
%A mi familia y amigos que

\hspace{3mm}
%me han apoyado incondicionalmente estos cuatro años.

\bigskip

\hspace{3mm}
%Agradecer también a los directores de este Trabajo de Fin de Grado,

\hspace{3mm}
%Eduardo Segredo y Carlos Segura,

\hspace{3mm}
%por la ayuda en el desarrollo del mismo

\hspace{3mm}
%y por adentrarme en el mundo de la investigación.

\end{large}

}

%%%%%%%%%%%%%%%%%%%%%%%%%%%%%%%%%%%%%%%%%%%%%%%%%%%%%%%%%%%%%%%%%%%%%%%%%%%%%%%%%
\newpage

\begin{huge}
License
\end{huge}
\begin{center}
\includegraphics[scale=1.5]{images/by-nc-sa_88x31}\\[10mm]
{\Large \copyright~This work is under Creative Commons license.
}
\end{center}



%%%%%%%%%%%%%%%%%%%%%%%%%%%%%%%%%%%%%%%%%%%%%%%%%%%%%%%%%%%%%%%%%%%%%%%%%%%%%%%
\newpage  %\cleardoublepage

%%%%%%%%%%%%%%%%%%%%%%%%%%%%%%%%%%%%%%%%%%%%%%%%%%%%%%%%%%%%%%%%%%%%%%%%%%%%%%%
\newpage  %\cleardoublepage
\begin{abstract}
{\em

With the raise of diseases related with unhealthy lifestyles such as heart-attacks, overweight, diabetes, etc. Encouraging healthy and balanced patterns in the population is one of the most important action points for governments around the world. Furthermore, it is actually even a more critical situation when a high percentage of patients are child and teenagers whose habits consists merely in eating fast food or ultra-processed food and a sedentary life.

The development of healthy and balanced menu plans became a routine task for physicians and nutritionists, and it is at this point that computer science has taken an important role. Discovering new approaches for generating healthy and balanced as well as inexpensive menu plans will play a part in banish of diseases from actual and new generations.

In this work, a decomposition algorithm called Multi-objective Evolutionary Algorithm Based on Decomposition have been developed in order to compared the performance of the algorithm for solving the menu-planning problem against some state-of-art evolutionary algorithms such as Non-dominated Sorting Genetic Algorithm II and Strength Pareto Evolutionary Algorithm.

In order to evaluate the performance of the developed algorithm, an exhaustive experimental research was made. Firstly, we focused on evaluate the parameter tuning of the algorithm so afterwards the best configuration found could be compared with other algorithms.  

}

\begin{keywords}
menu-plan, computer-science
\end{keywords}

\end{abstract}

%%%%%%%%%%%%%%%%%%%%%%%%%%%%%%%%%%%%%%%%%%%%%%%%%%%%%%%%%%%%%%%%%%%%%%%%%%%%%%%
\newpage{\pagestyle{empty}}
\thispagestyle{empty}

%%%%%%%%%%%%%%%%%%%%%%%%%%%%%%%%%%%%%%%%%%%%%%%%%%%%%%%%%%%%%%%%%%%%%%%%%%%%%%%


\pagestyle{myheadings} %my head defined by markboth or markright
% No funciona bien \markboth sin "twoside" en \documentclass, pero al
% ponerlo se dan un montón de errores de underfull \vbox, con lo que no se
% ha puesto.
\markboth{Alejandro Marrero Díaz}{On the healthy and balance menu planning automatisation}

%%%%%%%%%%%%%%%%%%%%%%%%%%%%%%%%%%%%%%%%%%%%%%%%%%%%%%%%%%%%%%%%%%%%%%%%%%%%%%%
%Numeracion en romanos
\renewcommand{\thepage}{\roman{page}}
\setcounter{page}{1}

\tableofcontents
%%%%%%%%%%%%%%%%%%%%%%%%%%%%%%%%%%%%%%%%%%%%%%%%%%%%%%%%%%%%%%%%%%%%%%%%%%%%%%%
\newpage{\pagestyle{empty}}
\listoffigures
%%%%%%%%%%%%%%%%%%%%%%%%%%%%%%%%%%%%%%%%%%%%%%%%%%%%%%%%%%%%%%%%%%%%%%%%%%%%%%%
\newpage{\pagestyle{empty}}
\listoftables
%%%%%%%%%%%%%%%%%%%%%%%%%%%%%%%%%%%%%%%%%%%%%%%%%%%%%%%%%%%%%%%%%%%%%%%%%%%%%%%
\newpage{\pagestyle{empty}}

\chapter{Motivation}\label{ref:motivation}
\section{Description of the Master thesis}

\section{Antecedent and current status of the topic}
\newpage{\pagestyle{empty}}
\chapter{Introduction}\label{ref:intro}
\section{Optimisation Problems}
An Optimisation Problem~(OP) is a problem which has a score function and bounds where the main task is to find a input that optimises the score function. Optimisation problems can be categorised as \textit{discrete optimisation problem~(DOP)} or \textit{continuous optimisation problem~(COP)} whether the variables of the problem are discrete or continuous. 
Formally speaking, an OP can be described as follows:
\begin{equation*}
min\;f(x), x \in \chi,\quad s.t.\:\Omega
\end{equation*}
where $\chi \subset\mathbb{R}^{n}$ is the search space defined over a set of \textit{n} decision variables  $x = ~(x_{1}, x_{2},..., x_{n})$, $f: \chi \rightarrow \mathbb{R}$ is the score function and $\Omega$ is the restrictions set in $x$. This is the definition for a minimisation DOP, although it would be equivalent for a maximisation DOP changing $min\;f(x)$ by $max\;f(x)$. The same happens for a COP, if in addition the decision variables are set over $\mathbb{Z}^{n}$ instead of $\mathbb{R}^{n}$ .

Furthermore, optimisation problems may have more than one score function and they are called \textit{Multi-Objective Optimisation Problems~(MOOP)}. This is the primary field of study in this work so, hereinafter all the references to optimisation problems in this work will be to MOOP.
\subsection{Multi-Objective Optimisation Problems}

Multi-Objective Optimisation Problems~(MOOPs) are optimisation problems which have two or more objective functions to optimise and those objective functions can take opposite directions (thinking about directions as \textit{minimise or maximise}). Besides, a MOOP can discrete or continuous considering whether the variables of the problem are discrete or continuous. \\
Formally speaking, a MOOP can be described as finding a vector \textbf{$x$} inside the problem's search space \textit{$\chi$} in such way that optimises the vector of objective functions \textit{$f(x)$}\cite{search}:
\begin{align*}
min\;f(x) & = (f_{1}(x), f_{2}(x), ..., f_{k}(x)), \: x\;\in\chi \\
 g_{i}(x) & \leq 0, \: i = 1, 2, ..., q. \\
 h_{i}(x) & \leq 0, \: i = 1, 2, ..., p.
\end{align*}
where $x = (x_{1}, x_{2}, ..., x_{n}) \in \mathbb{Z}^{n}$, are the objective functions to optimise $f_{i}: \mathbb{Z}^{n} \rightarrow \mathbb{R}, \; i = 1, ..., k$ being \textit{n} the number of decision variables and $g_{i}: \mathbb{Z}^{n} \rightarrow \mathbb{R}, \; i = ~1, ..., q$ and $h_{i}: \mathbb{Z}^{n} \rightarrow \mathbb{R}, \; i = 1, ..., p$ are the problem's restriction functions.

Moreover, the standard method for solving a MOOP is the well-known \textit{Pareto method}\cite{search}. The Pareto method is based on the non-dominance principle\cite{search, metaheuristics}. On the one hand, dominance means that given two solutions for a MOOP, one solutions dominates the other one when it has as least the same quality for every objective and, it has strictly more quality for one of them than the other solution. Formally, this can be expressed as follows\cite{search}:
\begin{align*}
A \succeq  B \Leftrightarrow & \forall i \in \{1, 2, ..., n\} \: a_{i} \leq b_{i}, \\
& y\;\exists i \in \{1, 2, ..., n\}, \: a_{i} < b_{i}
\end{align*}

On the other hand, it is the direction conflict between objectives which leads to solutions with trade-offs between those objectives. So, at this point it is where the \textit{non-dominance} appears. The non-dominance refers the situation where a solution it is not dominated by any other solution of the problem. That means that it can not be found any other solution to the problem which increases the quality of any objective without irredeemably decreases the quality of another one. Non-dominated solutions may be found at the limits of the search space (\textit{$\chi$}) and are those which shape the Pareto set.
%%%%%%%%%%%%%%%%%%%%%%%%%%%%%%%%%%%%%%%%%%%%%%%%%%%%%%%%%%%%%%%%%%%%%%%%%%%%%%%%%%%%
\newpage
\section{Evolutionary Algorithms}

Nowadays, there are many methods for solving MOOPs but they can be classified merely in two types of methods: \textit{approximated methods and exacts methods}. The different categories of exact and approximated methods can be seen down below.

\begin{figure}[!ht]\label{opt_met}
\centering
\includegraphics[width=0.8\textwidth]{mem/images/meta.png}
\caption{Optimisation methods}
\end{figure}

On the one hand, \textit{exacts methods} are those which ensure that, if there is an optimal solution to the facing problem they will be able to find it. However, even though these methods guarantee reaching the optimal solution they have a important drawback on its performance. Assuring the optimal solution implies increasing the computational work and hence more time to obtain the solution.

On the other hand, \textit{approximated methods} are very popular nowadays even though they do not guarantee reaching the optimal solution for a problem. Nevertheless, approximated methods can obtain high quality solutions in an assumable time due they set a balance between computational performance and solution quality. 

Approximated methods can be divided in two categories: \textit{heuristic algorithms} and \textit{meta-heuristics algorithms}. However, this work it is focus primarily in \textit{meta-heuristics algorithm} and even more specifically in the field of \textit{evolutionary algorithms}.

Evolutionary algorithms~(EA) develop the metaphor of natural evolution, the survival of the fittest individual\cite{eiben}. This is, given a population of individuals in some environment with limited resources, the competition for surviving causes natural selection and the fittest individuals are more likely to survive and reproduce. Nevertheless, there are several variants of EA and they can be classified as:
\begin{enumerate}
    \item Genetic Algorithms\cite{Whitley1994, Algorithms2004, Sivanandam2008}.
    \item Evolutionary Strategies\cite{Beyer2002, Hansen2017}.
    \item Differential Evolution\cite{Algorithm2006, DE1, DE2, DE3}.
\end{enumerate}

Moreover, there are a specific group of evolutionary algorithms for solving multi-objective problems known as \textit{Multi-Objective Evolutionary Algorithms~(MOEAs)}. MOEAs can be classified in many different subgroups considering the main approach underlying the algorithm\cite{ZHOU201132}:
\begin{itemize}
    \item MOEAs based on decomposition, i.e. MOEA/D\cite{Zhang2007, Ma2018}.
    \item MOEAs based on preference, i.e. NSGA-II\cite{996017}.
    \item Indicator-based MOEAs, i.e. IBEA\cite{IBEA}.
    \item PSO-based approaches, i.e. SPEA-2\cite{Laumanns2001SPEA2}.
\end{itemize}

Considering the field of optimisation problems, the natural evolution metaphor is develop as follows:
\begin{enumerate}
    \item The problem to solve and its bounds is the environment with limited resources.
    \item A set of random initial solutions for the given problem are the first individuals at generation zero. 
    \item The population of individuals reproduce between each other applying genetic operators to generate offspring. Commonly combination and mutation.
    \item At each generation, the individuals within a population compete and the fittest individuals (the better quality solutions) survive.
    \item Steps three and four are repeated until reaching the stop condition.
\end{enumerate}
Generally, the before metaphor can be shown as a pseudocode\cite{eiben}:

\begin{algorithm}[H]
\begin{algorithmic}[1]
  \State INITIALISE population with random candidate solutions
  \State EVALUATE each candidate
  \While{not StopCriteria satisfied}
    \State SELECT parents
    \State RECOMBINE pairs of parents
    \State MUTATE the result offspring
    \State EVALUATE new candidates
    \State SELECT individuals for the next generation 
  \EndWhile
  \State \textbf{end}
  \end{algorithmic}
  \caption{Pseudocode of an EA.}
\end{algorithm}


%%%%%%%%%%%%%%%%%%%%%%%%%%%%%%%%%%%%%%%%%%%%%%%%%%%%%%%%%%%%%%%%%%%%%%%%%%%%%%%%%%%%%%%
\newpage
\section{Menu Planning Problem Formulation considered}

\begin{comment}
The Menu Planning Problem \textit{(MPP)} is a well-known NP-Problem which has been trying to computerise since 1960\cite{Ngo2016}. In essence, the MPP is to find a set of dishes combination which satisfies some restrictions of budge, variety and nutritional requirements for a \textit{n} days sequence. In addition, it can include other constraints such as user preferences, cooking time or the number for meals each day.
Even though there is not consensus about the number of objectives that a MPP's formulation may have, in almost every formulation the cost of the menu plan is considered as one of the main objectives to optimise\cite{Ngo2016, Moreira2018} but it also supports other objective functions like maximising the variability or minimising the cooking time.

Furthermore, the MPP can be studied as a multi-objective problem\cite{Seljak2009} if the amounts of nutrients requirements and cost of the meals are considered independent objectives. This approach leads to reduce the MPP to a Multi-dimensional Knapsack Problem \textit{(MDKP)} where the maximum amount of each nutrient define the limit of the multiple dimensions. However, the MPP is also studying as a single-objective problem where mainly research define the objective function as the total cost of the meals. For instance, a single-objective approach for the MPP is in\cite{Moreira2018} where the authors proposed an evolutionary approach to solving the 5-day Single-Objective Menu Planning Problem composed by three meals daily, using as a function to minimise the total cost of the designed menus. In addition, the set of constraints that the researchers defined to this problem are moderately different from the usual constraints set for the typical MPP. In this occasion, the authors set the student age group, the school category, school duration time, school location, variety of preparations, the maximum amount to be paid for each meal and finally, the lower and upper limits of macro-nutrients as the constraints set to be satisfied for each solution to be considered feasible. Within this research, the authors used the generic Genetic Algorithm (GA) for the computational experiments. The results obtained from the generic GA where compared with a Greedy-based approach. The results prove that the GA outperforms the Greedy-based approach when the limit values of the meals are fixed at R\$ 2.00 for breakfast, R\$ 4.00 for lunch and R\$ 2.00 for the snack. (BRL - R\$ 1.0 ~~ USD - \$ 0.31).

In the other hand, in the paper\cite{Funabiki2011}, the authors refer to the Two-phase Cooking N-day Menu Planning Problem where the objective is to maximise the preferences among the selected foods in the menu plan. The conditions which shape the set of constraints that must be satisfied are only three. The total cooking time of any day must not exceed the limit specified, only foods that which allow two-phase cooking can be selected for two-phase cooking and finally, the food cannot be repeated more times than a certain repeat constraint. In order to face this problem, the researchers used a simple greedy method prioritising the user-specified preference with the cooking time of each food.

Eventually, another study where the MPP is faced as a single-objective problem is\cite{Sufahani2014}. Here, the authors set up a mathematical model to solve the MPP considering only one objective function. The model's goal is to minimise the budget provided by the government subject to the restriction of trying to maximise the variety of dishes. Furthermore, the model tries to create menus in such a way they maximise the nutritional requirements. For the computational experiments, the researchers programmed an Integer Programming algorithm in Matlab using LPSolve. Furthermore, the given results, taking into account that the optimal solution was found within one second, are better compared to other heuristics like GA.

As can be seen, there is a certain variety within the optimisation methods for solving the single-objective MPP approach. Despite that, Evolutionary Computing \textit{(EC)} techniques, such as GA, are mostly cited in the related bibliography as a good choice\cite{Ngo2016, Seljak2009, Moreira2018}. 
\end{comment}
In this particular case, a new formulation of the Menu Planning Problem proposed in \cite{Miranda2018} for school cafeterias is considered. The authors defined two objectives: meal cost and variety of dishes.
On the one hand, as usual in MPP, one goal is to minimise the total cost of the meal plan generated. Since the meal plan is designed for school cafeterias, the authors considered three meals in each menu: first course, second course and dessert. Formally, the meal plan cost can be defined as follows:
\[
    min\; C = \sum_{i=1}^{n}{c_{fc_{i}} + c_{sc_{i}} + c_{d_{i}}}
\]
where C is the total cost of the menu plan and $c_{fc_{i}}, c_{sc_{i}}, c_{d_{i}}$ represent the cost of the first course, second course and dessert in $n$ days.

On the other hand, an assorted menu plan is a must for children in order to avoid them to annoy about the food. For that reason, the second objective is to minimise the level of repetition of dishes and food groups in a certain menu plan. 
\[
    min\; L_{Rep} = \sum_{i=1}^{n}{v_{table_{i}} + \frac{p_{fc}}{d_{fc_{i}}} + \frac{p_{sc}}{d_{sc_{i}}} + \frac{p_{ds}}{d_{ds_{i}}} + v_{FG_{i}}}
\]
Where $L_{Rep}$ is the level of repetition to minimise, $v_{table_{i}}$ represents the compatibility between the courses $c_{fc_{i}}, c_{sc_{i}}, c_{d_{i}}$ for day $n$, $p$ is a penalty constant for every kind of course and $d$ stands for the number of days since the course was repeat for the last time. Finally, $v_{FG_{i}}$ is the penalty value for repetition of food groups in the last five days.

Additionally, in order to consider a menu plan as feasible, it must keep to some restrictions about a set nutritional requirements ($N$). The nutritional requirements considered in this formulation are the follow:
\begin{AutoMultiColItemize}
    \item Folic acid.
    \item Calcium.
    \item Energy (Kcal).
    \item Phosphorus.
    \item Fat.
    \item Iron.
    \item Magnesium.
    \item Potassium.
    \item Protein.
    \item Selenium.
    \item Sodium.
    \item Vitamin A.
    \item Vitamin B1.
    \item Vitamin B2.
    \item Vitamin B6.
    \item Vitamin B12.
    \item Vitamin C.
    \item Vitamin D.
    \item Vitamin E.
    \item Iodo.
    \item Zinc.
\end{AutoMultiColItemize}
There is also a vector $R$ in which the minimum and maximum amount of each nutritional requirement is stored. So formally speaking, a menu plan is feasible only if
\[
    \forall	 n \in N\; : R_{min_{n}} \leq I_{n} \leq R_{max_{n}}
\]
where $I_{n}$ is the amount of the n-nutritional requirement in the menu plan.

Lastly, the set $G$ of food groups considered for the available meals is: 
\begin{AutoMultiColItemize}
    \item Meat.
    \item Cereal.
    \item Fruit.
    \item Dairy.
    \item Fish.
    \item Vegetable.
    \item Shellfish.
    \item Legume.
    \item Pasta.
    \item Others.
\end{AutoMultiColItemize}


\newpage{\pagestyle{empty}}
\chapter{Algorithms}\label{ref:algorithm}
\section{Multiobjective Evolutionary Algorithm Based on Decomposition}
Multiobjective Evolutionary Algorithm Based on Decomposition (MOEA/D) is an evolutionary algorithm for multiobjective optimisation proposed by Qingfu Zhang and Hui Li in 2007\cite{Zhang2007}. The underlying idea behind this algorithm is to decompose a multiobjective optimisation problem into a number of scalar optimisation sub-problems and optimises them simultaneously. It also harnesses in the well-known feature of Pareto optimal solutions to a MOP, which sustains that an optimal solution for a scalar optimisation problem with an objective function as the aggregation of all the $f_{i}$ could be the same as the Pareto optimal solution for the MOP\cite{Zhang2007}.

The decomposition approach of MOEA/D takes place where the algorithm decomposes a MOP into \textit{N} sub-problems and simultaneously optimises every single sub-problem at each generation. Furthermore it establish some relations between sub-problems and organised them in neighbourhoods. These neighbourhoods are shaped by sub-problems which coefficient vectors are very similar to each other and every single sub-problem is optimised based on its neighbouring sub-problems information. Therefore, the optimal solution for two neighbouring sub-problems should be very similar\cite{Zhang2007}.

On the other hand, the process of decompose a MOP into \textit{N} sub-problems can be done from some different approaches. However, as the authors referred\cite{Zhang2007} it this paper the MOEA/D uses the Tchebycheff Approach\cite{Ma2018} to decompose a MOP. Formally, the Tchebycheff Approach it is defined as follows:
\[
min\quad g^{te}(x|\lambda,z^{*}) = max_{i=1}^{m}\{\lambda_{i}|f_{i}(x)-z_{i}^{*}|\}
\]

where $z^{*} = (z^{*}_{1}, ..., z^{*}_{m})$ is the reference point with the best solution founds so far for each sub-problem and $\lambda_{i} = (\lambda_{i,1}, ..., \lambda_{i,m})$ is a even spread weight vector for each sub-problem \textit{i}.

In addition, MOEA/D version implemented in this paper works as follows. It takes a MOP, the population size, a stopping criterion and the number of neighbours for each neighbourhood. The number of sub-problems in this implementation are the MOP's objectives. Then, it starts by randomly generate \textit{N} even spread weight vectors and compute the Euclidean distance between each others to shape the neighbourhoods, generates an initial random population and computes the reference point $Z^{*}$. After the initialisation phase, it goes into the main loop where, until the stopping criteria is not satisfied, the algorithm preforms theses steps for each individual of the population:
\begin{itemize}
  \item Reproduction: generates a new child individual from two randomly selected neighbours \textit{l, k}.
  \item Improve: maintains the new child under the limits of the problem's search space.
  \item UpdateZ: updates the reference point by comparing it with each new child individual.
  \item Update Neighbours: if the new child individual performs better than any neighbours, replaces the neighbour with the brand new individual.
\end{itemize}
Finally, MOEA/D returns the PF's points found. \\

Concretely, the algorithm can be outlined as follows: \\

\begin{algorithm}[H]
  \KwData{MOP, PopSize, StopCriteria, Neighbours}
  \KwResult{PF}
  SetRandomWeightVectors()\;
  EuclideanDistance()\;
  GenerateRandomPopulation()\;
  InitializeZ()\;
  \While{not StopCriteria satisfied}{
    \For{each sub-problem do} {
      l,k = getRandomNeigbours()\;
      child = reproduce(l, k)\;
      child = improve(child)\;
      updateZ(child)\;
      updateNeighbouringSolutions(child)\;
    }
  }
  \caption{MOEA/D version of this thesis.}
\end{algorithm}
\section{Strength Pareto Evolutionary Algorithm}
\section{Non-dominated Sorting Genetic Algorithm II}


\newpage{\pagestyle{empty}}
\chapter{Results}\label{ref:results}
\input{chapters/results.tex}
\newpage{\pagestyle{empty}}
\chapter{Summary and conclusions}\label{ref:conclusions}
\section{Conclusions and future work}

As seen in Chapter \ref{ref:results}, NSGA-II still being the state-of-art in Multi-Objective Evolutionary Algorithms since it outperforms both SPEA-2 and MOEA/D with statistically significant differences for this recently proposed MPP formulation.

Regarding MOEA/D algorithm, the quite simple version developed for this Master Thesis does not obtain as high quality solutions as NSGA-II or SPEA-2. In the experimental evaluation explained in Chapter \ref{ref:results}, the population size and neighbourhood size seems not to have a high impact into the perfomance of the MOEA/D algorithm as it can be appreciated in the ranking from the preliminary experimental evaluation in Table \ref{ranking}. 

For further work, considering a new approach for initial weight generation may be a interesting choice as well as a more depth experimental evaluation with MOEA/D considering the mutation and crossover probability rate and increasing the evaluation limit to 4e8 evaluations.
 
\newpage{\pagestyle{empty}}
\chapter{Budget}\label{ref:budge}
At this point, it will be introduce the total budget necessary for the development of this project.

\section{Computer science engineer wage}
First of all, considering all the work done in this Master's thesis, the spent time in each activity and finally the cost of each hour of work, the total salary for a computer science engineer is 10.100\euro{}. The following table shows the spent time, the cost per hour and the total cost of all the activities done in this project.

\begin{table}[!h]
\centering
\resizebox{\textwidth}{!}{%
\begin{tabular}{|c|cc|c|}
\hline
\textbf{Work} & \multicolumn{1}{c|}{\textbf{Time Spent (hours)}} & \multicolumn{1}{l|}{\textbf{Cost/Hour}} & \multicolumn{1}{l|}{\textbf{Total Cost (\euro{})}} \\ \hline
Investigating about the state-of-art & \multicolumn{1}{c|}{50} & 30 & 1500 \\ \hline
Developing the MOEA/D algorithm & \multicolumn{1}{c|}{50} & 25 & 1250 \\ \hline
Designing the computational experiments & \multicolumn{1}{c|}{70} & 35 & 2450 \\ \hline
Analysing the experiments's results & \multicolumn{1}{c|}{50} & 50 & 2500 \\ \hline
Writing the dissertation & \multicolumn{1}{c|}{80} & 30 & 2400 \\ \hline
{\color[HTML]{000000} \textbf{Total Cost}} & \multicolumn{1}{l}{} & \multicolumn{1}{l|}{} & \textbf{10.100} \\ \cline{1-1} \cline{4-4} 
\end{tabular}%
}
\caption{Description of the work done and its cost.}
\label{my-label}
\end{table}

\newpage
\section{Equipment}
On the other hand, it must be considered all the gear used in this project. Therefore, the cost of using and buying all the necessary gear must be added to the budget of this project.
\begin{table}[!h]
\centering
\begin{tabular}{cc}
\hline
\multicolumn{1}{|c|}{\textbf{Gear}} & \multicolumn{1}{c|}{\textbf{Cost (\euro{})}} \\ \hline
\multicolumn{1}{|c|}{MSI GS63 Stealth 8RD} & \multicolumn{1}{c|}{1500} \\ \hline
\multicolumn{1}{|c|}{BenQ GW2780 27"} & \multicolumn{1}{c|}{164.46} \\ \hline
\multicolumn{1}{|c|}{Logitech Wireless Mouse M185} & \multicolumn{1}{c|}{20} \\ \hline
\textbf{Total Cost} & \textbf{1684.46}
\end{tabular}
\caption{List of the gear used in this project.}
\label{my-label}
\end{table}

\section{Total cost}
Finally, take into account both gear and the computer science engineer wage, the total cost of this Master's thesis is $11784,46$ \euro{}.
\newpage{\pagestyle{empty}}
%
\addcontentsline{toc}{chapter}{Bibliography}
\bibliographystyle{plain}

\bibliography{memtfm}
%\nocite{*}

%%%%%%%%%%%%%%%%%%%%%%%%%%%%%%%%%%%%%%%%%%%%%%%%%%%%%%%%%%%%%%%%%%%%%%%%%%%%%%%

\end{document}

